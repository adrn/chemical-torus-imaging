% To-dos
% ------

\documentclass[modern]{aastex63}

\usepackage{xcolor}
\usepackage{amsmath}

\newcommand{\documentname}{\textsl{Article}}
\newcommand{\sectionname}{Section}
\renewcommand{\figurename}{Figure}
\newcommand{\equationname}{Equation}
\renewcommand{\tablename}{Table}

% Misc.
\newcommand{\bs}[1]{\boldsymbol{#1}}

% Missions
\newcommand{\project}[1]{\textsl{#1}}

% Packages / projects / programming
\newcommand{\package}[1]{\textsl{#1}}
\newcommand{\acronym}[1]{{\small{#1}}}
\newcommand{\github}{\package{GitHub}}
\newcommand{\python}{\package{Python}}

% Stats / probability
\newcommand{\given}{\,|\,}
\newcommand{\norm}{\mathcal{N}}
\newcommand{\pdf}{\textsl{pdf}}

% Maths
\newcommand{\dd}{\mathrm{d}}
\newcommand{\TT}[1]{\ensuremath{{#1}^{\mathsf{T}}}}
\newcommand{\transp}{\ensuremath{^{\mathsf{T}}}}
\newcommand{\inv}[1]{{#1}^{-1}}
\newcommand{\argmin}{\operatornamewithlimits{argmin}}
\newcommand{\mean}[1]{\left\langle #1 \right\rangle}

% Non-scalar variables
\renewcommand{\vec}[1]{\ensuremath{\bs{#1}}}
\newcommand{\mat}[1]{\ensuremath{\mathbf{#1}}}

% Unit shortcuts
\newcommand{\Msun}{\ensuremath{\mathrm{M}_\odot}}
\newcommand{\Mjup}{\ensuremath{\mathrm{M}_{\mathrm{J}}}}
\newcommand{\kms}{\ensuremath{\mathrm{km}~\mathrm{s}^{-1}}}
\newcommand{\mps}{\ensuremath{\mathrm{m}~\mathrm{s}^{-1}}}
\newcommand{\pc}{\ensuremath{\mathrm{pc}}}
\newcommand{\kpc}{\ensuremath{\mathrm{kpc}}}
\newcommand{\kmskpc}{\ensuremath{\mathrm{km}~\mathrm{s}^{-1}~\mathrm{kpc}^{-1}}}
\newcommand{\dayd}{\ensuremath{\mathrm{d}}}
\newcommand{\yr}{\ensuremath{\mathrm{yr}}}
\newcommand{\AU}{\ensuremath{\mathrm{AU}}}
\newcommand{\Kel}{\ensuremath{\mathrm{K}}}
\newcommand{\mas}{\ensuremath{\mathrm{mas}}}

% Astronomy
\newcommand{\DM}{{\rm DM}}
\newcommand{\abunratio}[2]{\ensuremath{{[\mathrm{#1}/\mathrm{#2}]}}}
\newcommand{\feh}{\abunratio{Fe}{H}}
\newcommand{\mh}{\abunratio{M}{H}}
\newcommand{\alphafe}{\abunratio{\alpha}{Fe}}
\newcommand{\mgfe}{\abunratio{Mg}{Fe}}
\newcommand{\df}{\acronym{DF}}
\newcommand{\logg}{\ensuremath{\log g}}
\newcommand{\Teff}{\ensuremath{T_{\textrm{eff}}}}
\newcommand{\mtwomin}{\ensuremath{M_{2, {\rm min}}}}

% Colors:
\definecolor{tabblue}{HTML}{4E79A7}
\definecolor{taborange}{HTML}{F28E2B}
\definecolor{tabgreen}{HTML}{59A14F}
\definecolor{tabred}{HTML}{E15759}
\definecolor{tabpurple}{HTML}{B07AA1}

% TODO:
\newcommand{\TODO}[1]{{\color{tabgreen}\textbf{TODO:} #1}}
\newcommand{\APWTODO}[1]{{\color{tabpurple}\textbf{APW TODO:} #1}}
\newcommand{\HOGGTODO}[1]{{\color{tabred}\textbf{HOGG TODO:} #1}}


% text macros
\newcommand{\methodname}{Abundance-moment foliation}
\newcommand{\method}{\acronym{TMR}}

% Project-specific macros - all others go in preamble.tex
\newcommand{\gaia}{\textsl{Gaia}}
\newcommand{\dr}[1]{\acronym{DR}#1}
\newcommand{\apogee}{\acronym{APOGEE}}
\newcommand{\sdss}{\acronym{SDSS}}
\newcommand{\sdssiv}{\acronym{SDSS-IV}}
\newcommand{\sdssv}{\acronym{SDSS-V}}
\newcommand{\galah}{\acronym{GALAH}}
\newcommand{\hermes}{\acronym{HERMES}}

\setlength{\parindent}{1.1\baselineskip} % trust in Hogg
\shorttitle{tags are separable from angles}
\shortauthors{apw, dwh, hwr, and more}

\begin{document}\sloppy\sloppypar\raggedbottom\frenchspacing % trust in Hogg
\graphicspath{ {figures/} }
\DeclareGraphicsExtensions{.pdf,.eps,.png}

\title{\textbf{%
Revealing the orbital structure of the Milky Way through the invariance
of element-abundance distributions}}


\newcommand{\affcca}{Center for Computational Astrophysics, Flatiron Institute, 162 Fifth Ave, New York, NY 10010, USA}
\newcommand{\affccpp}{Center for Cosmology and Particle Physics, Department of Physics, New York University, 726 Broadway, New York, NY 10003, USA}
\newcommand{\affmpia}{Max-Planck-Institut f\"ur Astronomie, K\"onigstuhl 17, D-69117 Heidelberg, Germany}
\newcommand{\affcolumbia}{Department of Astronomy, Columbia University, New York, NY 10027, USA}

\author[0000-0003-0872-7098]{Adrian~M.~Price-Whelan}
\affiliation{\affcca}

\author[0000-0003-2866-9403]{David~W.~Hogg}
\affiliation{\affcca}
\affiliation{\affmpia}
\affiliation{\affccpp}

\author{Kathryn~V.~Johnston}
\affiliation{\affcca}
\affiliation{\affcolumbia}

\author[0000-0003-4996-9069]{Hans-Walter~Rix}
\affiliation{\affmpia}

\author{others}

\begin{abstract}\noindent % Hogg strikes again
  The dynamics of the Milky Way---and nearby galaxies---are often
  understood in an approximation in which the gravitational potential
  is treated as simple and integrable and the distribution function
  time-invariant.
  Under these conditions, the Jeans equations are relevant and Jeans
  modeling is the standard---though not the only---methodology.
  Here we present an alternative approach for
  inference---\textsl{\methodname}---which makes
  use of the fact that there are generically gradients, or dependences
  of \emph{tags} (element abundnaces or stellar ages or other invariant stellar
  properties) on position and velocity in the Galaxy.
  The method exploits the fact that in steady state, the distribution of any tags
  must be invariant with respect to time-variable conjugate angles.
  Thus level sets in moments of the tag distribution must coincide with
  the orbital foliation of phase space: The tags directly reveal the orbit structure.
  As a demonstration of the concepts and methods, we use
  \apogee\ giant stars to look at vertical orbit structure in the Milky
  Way disk, and constrain the disk mass (holding the well-measured
  circular velocity fixed).
  We show that the disk mass can be constrained na\"ively at the
  XXX-percent level with this new method.
  We write down and discuss both frequentist-statistics and Bayesian
  forward-modeling approaches to inference.
  \methodname\ has great advantages:
  It does not depend on making measurements of second (or higher)
  moments of the velocity distribution, as Jeans modeling
  requires.
  It does not depend on potential separability, although it does require
  integrability.
  It can be written in a form that is, to first order, independent
  of the details of the stellar selection function.
  It does, however, require abundance (or other) gradients with respect to actions.
  Fortunately, God provides.
\end{abstract}

% \keywords{\raggedright
%  astrometry
%  ---
%  Galaxy:~disk
%  ---
%  Galaxy:~fundamental~parameters
%  ---
%  Galaxy:~kinematics~and~dynamics
%  ---
%  methods:~data~analysis
%  ---
%  stars:~abundances
%  ---
%  stars:~kinematics~and~dynamics
% }

\section*{}\clearpage

\section{Introduction}

Physicists of the 18th century worked out that the gravitational force
law in the Solar System is proportional to the inverse square of the
distance from the Sun (CITE Newton).
That inference was based on the observation that orbits in the Solar
System are closed ellipses, with the Sun at one focus, plus period
ratios of orbits at different mean distances (CITE Kepler).
We physicists of the 21st century are trying to work out the map of
gravitational accelerations (or, in some parlance, the gravitational
potential, or, in another, the mass distribution)  the Milky Way.
We are at a disadvantage, in some sense, because no-one gets to see
the orbits of stars in the Milky Way: Even in the whole history of
western science, stars have barely moved in Galactic coordinates.
The tools we use now generally ask statistical questions about the
orbit structure (CITE Jeans and Oort and B\&T); we never see the orbits directly.
The closest we come to seeing orbits directly in the Milky Way is in the
study of stellar streams (CITE), but these are rare only exist at large
Galactocentric distances.

Okay fine, really we are at a \emph{huge advantage} with respect to
our 18th century counterparts, because we have a very accurate theory
of gravity (CITE) and we have immense observational and computational
resources.
With these resources we have measured the element abundances of the
stellar photospheres of hundreds of thousands of stars (CITE) and
precise positions and velocities of millions.
In this \documentname\ we are going to demonstrate that these
measurements can be used to illuminate the orbit structure in the
Milky Way, even---or especially---in the dense, nearby parts of the
disk.
Our method will work only under strong assumptions, so we are
still in the business of revealing the orbits indirectly.
However, our approach is novel and there are many regimes in which it
will be more precise than any other method.

In a well-mixed population, stars are in a kinematic steady state:
As time goes on, stars move along their orbits, but they don't change their dynamical
invariants (by assumption!) and every location in angle (and by ``angle'' we mean the
coordinate conjugate to action in action--angle coordinates) is equally likely, or
equally probably populated in any snapshot.
The chemical-tagging insight (CITE) is that the stars also don't change
their surface element abundances as they orbit.
That is, in a well-mixed galaxy, the element abundances and the dynamical actions have
something in common:
They don't change with time, while the conjugate angles do.
This means that, for a well-mixed population, the detailed element abundances can only
be a function of actions, and never a function of conjugate angles.
That's a remarkably informative constraint on the configuration of the Milky Way in
the space of positions (3), velocities (3), and detailed abundances (10 to 30, depending
on survey).

Consider a collection of stars (localized, say, in phase space) for which we have
measured $D$ element abundances.
This collection of stars is drawn from some density or distribution in
the $D$-dimensional element-abundance space.
Now consider how this distribution varies as we move around in phase space.
It will vary in general (there are radial and vertical abundance gradients in the
disk, for example).
Any gradient we observe in these element-abundance distributions with respect to
phase-space coordinates (that is, the six-gradient) must not project
onto the directions of increasing (or decreasing) conjugate angles in phase space.
That is, all gradients with respect to phase-space coordinates
of the element-abundance distribution must be orthogonal to the
directions of increase (or decrease) of the conjugate angles, and lie in the subspace
of the directions of increase (or decrease) of the dynamical actions.
The trajectories of stars in the phase space (the dynamical tori) must lie along or
describe level surfaces in the element-abundance distribution!

HOGG: MOve to later? Or summarize?
...
One of the crazy things about all this is that any description of a general distribution in $D$-space
requires a \emph{lot of parameters}.
Even a trivial distribution---the Normal distribution---requires $0.5\,D\,(D+3)$ parameters for its
complete description, and anything more complex has more.
That means that there are a very large set of element-abundance statistics for which the
six-gradient exists or could be constraining for this project.
Not all of these parameters can be reliably measured in a finite data set,
let alone reliably seen to vary with phase-space location.
However, we choose to see this bug as a feature:
We have discovered, in effect, a \emph{combinatorically large set of constraints on the
dynamical actions and conjugate angles}.

WE WILL DEMONSTRATE WITH JUST VERTICAL KINEMATICS, BUT WE ARE NOT
GOING TO ASSUME SEPARABILITY.  Jeans models can suck it.

\section{Methodological generalities}

HOGG ASKS: IS THIS WRITTEN TOO PEDAGOGICALLY?

This project was motivated by plots of abundances of stars as a
function of vertical height $z$ and vertical velocity $v_z$ in
Galactocentric cylindrical coordinates, colored by element abundances.
Examples are shown in \figurename~APW.
In these plots,
the eye is drawn to \emph{abundance gradients}: The stars at small
heights and small vertical velocities have different abundance ratios,
on average, than stars at large heights and large vertical velocities.
But these positional and velocity gradients are related:
Stars at large absolute vertical velocities $v_z$ will, in the future, at some
times be at large absolute vertical positions $z$ (far from the plane, that is),
and stars at large absolute vertical positions will, in the future,
at some times be at large absolute vertical velocities.
That is, the stars will orbit in the Galaxy.
And stars don't---to very high precision---change their abundances as
they orbit.

One consequence of these observations is a new method for inferring
the orbit structure of the Milky Way:
If two small neighborhoods in phase space lie on the same orbit---that
is, they correspond to the same dynamical actions or they can be reached
by common initial conditions---they must contain stars with the same distribution of
element abundances.
This prediction depends on many detailed assumptions, such as that
the Galaxy is (approximately) phase mixed, and that the potential is
(approximately) time invariant and integrable.
The usefulness of this prediction for inference
depends on the existence of gradients: If there aren't
element-abundance-ratio gradients, there will be no information to work
with.

In more mathematical language: The orbits (which are 3-tori in
6-dimensional phase space) in a steady-state, integrable, phase-mixed
galaxy will be level hyper-surfaces in phase space of any moments or
statistics of the element-abundance distribution... HOGG

This point is illustrated graphically in \figurename~XXX:
In the fiducial model, the two overlaid orbits more-or-less follow something
like a mean abundance contour.
In the model in which the disk is made less massive (and the halo more massive
to keep the circular velocity constant), the orbits change shape: There is more
positional extent to an orbit relative to it's velocity extent.
The orbital frequency $\omega$ is, to order of magnitude, the velocity amplitude
divided by the spatial amplitude, and it is, to order of magnitude, $\sqrt{G\,\rho}$,
where $\rho$ is the mean enclosed mass density.
If stars were traveling on these lower-disk-mass orbits, they would have to
obtain higher abundances when they are passing through the disk midplane,
and lower abundances when they are at their greatest absolute vertical heights,
which is absurd: Stars do not change their abundances as they orbit.

We make this point clear in a different way in \figurename~YYY, in which
we have transformed into action--angle coordinates in phase space (CITE)
and we show scatterplots of the abundances as a function of conjugate
angles $\theta_z$, conjugate to the vertical actions $J_z$, for stars near
the orbits plotted in \figurename~XXX.
Using the fiducial potential to convert to actions and angles,
the abundance distribution is similar at different
angles $\theta_z$, or on different parts of the orbit.
Along the orbits generated by the low-mass-disk and high-mass-disk potentials,
the abundance distribution shows clear $m=2$ (or $\exp i\,2\,\theta_z$) dependences
on vertical angle $\theta_z$.
That is, just by plotting the abundances versus angle, we can make
inferences about the mass of the disk, or---more generally---the
gravitational potential of the Milky Way.

In order to make the abundance--angle plots in \figurename~YYY, we had
to DO SOME THINGS because we needed to visualize the global dependence
on angle using all the stars on all different kinds of orbits.
Because different orbits have different abundance-ratio
distributions, this is not a trivial transformation of the data.
In the inferences we do in \sectionname~AAA below, we
will use regression (frequentist) or forward modeling (Bayes) to
combine the data from stars on different orbits.

...The point that this works at one to three dimensions. It requires
abundance gradients, but not separability!  Of course you might not
have abundance gradients in all three action directions.

...The point that the method is combinatoric in abundance data.

...The point that this would be more complex in chaotic regions, and
maybe effectively inapplicable?

...The point that this doesn't depend on selection, provided that
stellar properties don't vary strongly with abundnaces, and that
abundance-measurement biases don't depend strongly on stellar
parameters. Both assumptions wrong in detail.

... The point that there will be frequentist statistics or
optimizations, and also Bayesian approaches.  The point that Bayes and
frequentism look very different here. Call out to orbital roulette,
and to Bovy et al.

In what follows, a very limited project...

\section{Data}

HOGG SAY: This section is APW's problem (ish).

APW: \apogee\ bread-and-butter and citations.

APW: Removal of stellar clusters from the data.

APW: Distance estimation.

APW: Choice of abundances.

APW: Final data properties (and possibly figures).

\section{Milky Way mass model and actions}

HOGG SAY: Possible section for APW to put details of the Milky Way model,
how we parameterize the disk mass, and how we compute actions and angles.
Or, if there isn't much to say, this can be part of the inferences section
below.

This section could describe our one mass-model parameter, and show how
the mass distribution and orbit structure vary as we vary it.

This section could also describe all the computational tips and tricks.

\section{Assumptions}

\methodname flows from a specific set of assumptions.
That is, we are going to make hard (and sometimes controversial) assumptions.
Our position is not that these assumptions are correct.
Our position is that our
methods are conditionally correct, conditioning on these assumptions.
\begin{description}
\item[integrable] HOGG: There are 3 invariants and 3 angles. This could perhaps be
  relaxed, but we don't know how yet.

\item[well mixed] HOGG: All angles equally likely.
  The well-mixed assumption will be violated substantially in the data;
  we will discuss this more below. But this is the fundamental assumption of
  the vast majority of inferences of the Milky Way mass distribution.

\item[selection] HOGG: Selection depends on position in the Milky Way, but not
  on element abundances. Really it could even depend on velocity! But it can't depend
  on abundances. That might be very slightly wrong.

\item[kinematic measurements] HOGG: Very strong assumption that phase space positions
  in 6-d are accurate and precise. So precise we don't have to generate them, we can
  condition on them!

\item[abundance measurements] HOGG: Assumption that stars in different parts of phase
  space get the same abundance measurements. Questionable in certain kinds of samples.
  HOGG: Oddly I don't think we need to assume that the chemical measurements have good
  (or any) noise estimates. They could be terrible, I think.

\item[smooth gradients] HOGG: Assumption about how the abundances can
  depend on action. Relatedly, the use of action and not log-action
  or zmax or whatever.
\end{description}

\section{Inferences}

HOGG: There are two approaches, with very different appearances, but
embodying the same fundamental assumptions.

HOGG: discriminative methods: frequentist forms; About finding the mass model in which
there is no residual dependence of abundance distributions on angles.
Technically these models are not strictly ``discriminative'' but it is a related
concept. Oh wait, I could truly make a discriminative model...

HOGG: generative methods: frequentist or Bayesian forms; About generating the abundances
with a model using only Actions.

HOGG: How are these qualitatively different methods related?

HOGG: Get more specific about forms and math.

HOGG: Show results on the data.

\section{Relationship to Jeans and other methods}

HOGG ASKS: What to write here? How detailed should we be? Or should this just
go to the discussion section?

\section{Discussion}

HOGG: What did we find? How do our conclusions differ from those before us?

HOGG: In particular, what can we say about the dark disk etc?

HOGG: How is what we did better than what came before or happens by other methods?

HOGG: What are the limitations of what we did; what did we sacrifice for our benefits?

HOGG: Or we could phrase this as a set of failure modes?

HOGG: Return to the assumptions; do any need more discussion? One that does is the smooth
gradients assumption: If we choose different invariants to use, we get slightly different
answers, because (HOGG presumes) that the functional form for the abundance means fits
better or worse. So these results aren't definitive; we should probably use a more flexible
model for the abundance gradients, like a GP or etc.

HOGG: Another is the good abundances and selection assumptions: Do the abundances affect
the selection, and are the abundance measurements a function of stellar type? Either way,
we will inherit biases.

HOGG: What is the limit of this method as we go forward: Many abundance ratios? Many other
statistics of the abundance distribution? How do things change if the abundances track
each other exactly? Is that a problem? Note that we scale much better than chemical tagging
here.

\acknowledgments
It is a pleasure to thank
  Melissa Ness (Columbia),
  who started the conversation that started this project, many years ago.
We also thank
  Jo Bovy (Toronto),
  Anna-Christina Eilers (MIT),
  Suroor S. Gandhi (NYU),
  David Spergel (Flatiron),
  Eugene Vasiliev (Cambridge),
  and the Dynamics and Astronomical Data groups at the Flatiron Institute
for valuable discussions and input.
DWH was partially supported by HOGG GRANT DETAILS.
AMPW was partially supported by HOGG GRANT DETAILS.
HWR was partially supported by HOGG GRANT DETAILS.
This research was conducted in part at the Aspen Center for Physics,
which is supported by National Science Foundation grant \acronym{PHY-1607611}.

SDSS or whatever SPECTROSCOPY ack?

This work has made use of data from the European Space Agency (\acronym{ESA}) mission
\gaia\ (\url{https://www.cosmos.esa.int/gaia}), processed by the \gaia\ Data
Processing and Analysis Consortium (\acronym{DPAC},
\url{https://www.cosmos.esa.int/web/gaia/dpac/consortium}). Funding for the
\acronym{DPAC}
has been provided by national institutions, in particular the institutions
participating in the \gaia\ Multilateral Agreement.

% \facilities{
% \gaia,
% \galah,
% \hermes
% }

% \software{
% \code{Astropy} \citep{astropy, astropy2},
% \code{corner} \citep{corner},
% \code{emcee} \citep{emcee},
% \code{IPython} \citep{ipython},
% \code{matplotlib} \citep{matplotlib},
% \code{numpy} \citep{numpy},
% \code{pyia} \citep{pyia},
% \code{scipy} \citep{scipy}
% }

\end{document}
